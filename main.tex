\documentclass{article}

% if you need to pass options to natbib, use, e.g.:
%     \PassOptionsToPackage{numbers, compress}{natbib}
% before loading neurips_2021

% ready for submission
\usepackage[preprint]{neurips_2021}

% to compile a preprint version, e.g., for submission to arXiv, add add the
% [preprint] option:
%     \usepackage[preprint]{neurips_2021}

% to compile a camera-ready version, add the [final] option, e.g.:
%     \usepackage[final]{neurips_2021}

% to avoid loading the natbib package, add option nonatbib:
%    \usepackage[nonatbib]{neurips_2021}

\usepackage[utf8]{inputenc} % allow utf-8 input
\usepackage[T1]{fontenc}    % use 8-bit T1 fonts
\usepackage{hyperref}       % hyperlinks
\usepackage{url}            % simple URL typesetting
\usepackage{booktabs}       % professional-quality tables
\usepackage{amsfonts}       % blackboard math symbols
\usepackage{nicefrac}       % compact symbols for 1/2, etc.
\usepackage{microtype}      % microtypography
\usepackage{xcolor}         % colors

\title{Investigation of the average song length and valence over time due to the rising popularity of music streaming services}

% The \author macro works with any number of authors. There are two commands
% used to separate the names and addresses of multiple authors: \And and \AND.
%
% Using \And between authors leaves it to LaTeX to determine where to break the
% lines. Using \AND forces a line break at that point. So, if LaTeX puts 3 of 4
% authors names on the first line, and the last on the second line, try using
% \AND instead of \And before the third author name.

\author{%
  Johannes Gaus\\
  Matrikelnummer 5302275\\
  \texttt{johannes.gaus@student.uni-tuebingen.de} \\
  \And
  Tobias Ziefle\\
  Matrikelnummer 5794678\\
  \texttt{tobias.ziefle@student.uni-tuebingen.de} \\
}

\begin{document}

\maketitle

\begin{abstract}
 We are planning to use the collection of  \href{https://www.kaggle.com/rodolfofigueroa/spotify-12m-songs}{Spotify 1.2M+ Songs} to see how the general length of songs has changed over time due to the rising popularity of music streaming services and their payment guidelines.\\
 We also want to investigate how the valence of the songs changes over time.\\  
 We are planning to visualize our data with plots to show the developement of the average songlength over time and  the development of the valence.
\end{abstract}

\newpage

\section{data set}
The data set, that is investigated within this paper originates from Rodolfo Figureroa and was gathered with the Spotify Developer API. The API allows researchers to crawl through the Spotify song database, which includes more than 70 million songs. Our dataset consists of 1.2 million of those songs, that have been preselected by the dataset's author.\\
To do so, they used the music encyclopedia \glqq Musicbrainz\grqq. The songs within our dataset have been released between 1900 and the 18th of December in 2020. It consists of 24 columns, that hold information like track title, artist, and album title. The following columns are from special interest, for our studies:\\
track length [mms], release year[YYYY] and valence [floating point between 0 and 1].

\section{question}
Media reports state [Zitat BR], that music is getting shorter on average. The question answered within this paper is, whether those statements can be proven by data and whether there is an interconnection to Spotify's rising market shares of online music streaming. \\
Additionally, the paper investigates, whether songs have gotten sadder over time, as media reports indicate. [https://www.insidescience.org/news/popular-music-getting-sadder-and-angrier-new-study-finds] 

\section{visualization/plots}

\section{question answer}


\section{limitations/confidences/problems}
The dataset used suffers from bias as it does not contain all songs that have ever been released. For datasize reasons, using the whole dataset is not practicle for our research, as its analysis would exceed our computing ressources. The actual average song length therefore may be different. The underlying MusicBrainz library may favour some music genres, whith overall longer or shorter track lengths. Additionally, the database is eddited by its users and is no subject to any moderation or quality assurance. Therefore there is no assurance, that the tracks in the dataset are spread equally over all music genres. 

Regarding the papers second research question, whether music get sadder, we fully depend on the Audio features generated by the Spotify developer API. How those features are determined is not clearly specified throughout the corresponding API reference. Happiness and sadness are subjective impressions and perceived differently. The valance feature of our dataset is generated through algorithms and therefore biased by their authors. 
\section{Citations, figures, tables, references}

\section{support your argument with citations}

\section*{References and Git Repo}


\end{document}
